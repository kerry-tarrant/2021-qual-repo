\documentclass{book}
\usepackage{indentfirst}
\usepackage{amsmath}
\usepackage{amsfonts}
\usepackage{latexsym}
\usepackage{graphicx}
\usepackage{comment}
\usepackage{tikz}
\usetikzlibrary{arrows.meta,quotes}
\usepackage{hyperref}

\usepackage{algorithmic}
\usepackage{algorithm2e}

\newtheorem{theorem}{Theorem}

\newcommand{\latex}{\LaTeX\xspace}

\renewcommand{\arraystretch}{1.5}

\usepackage[]{amsthm}
\usepackage[]{amssymb}
\usepackage{enumitem}

\title{Analysis Notes}
\author{Kerry Tarrant}
\date\today

\begin{document}
\maketitle
\tableofcontents{}


\chapter{Measure Theory}

\section{Preliminaries}
Note: Understand the analogy between measuring "volume" and the concept of this section.

Note: Every open set in $\mathbb{R}$ is a countable union of disjoint open intervals. Every open set in $\mathbb{R}^{d}$, $d\geq 2$, is almost the disjoint union of closed cubes. Almost means that only the boundaries of the cubes can overlap.


Point


\end{document}