\documentclass{book}
\usepackage{indentfirst}
\usepackage{amsmath}
\usepackage{amsfonts}
\usepackage{latexsym}
\usepackage{graphicx}
\usepackage{comment}
\usepackage{tikz}
\usetikzlibrary{arrows.meta,quotes}

\usepackage{algorithmic}
\usepackage{algorithm2e}

\newcommand{\latex}{\LaTeX\xspace}

\renewcommand{\arraystretch}{1.5}

% More math packages
\usepackage[]{amsthm}
\usepackage[]{amssymb}
\usepackage{enumitem}

% Symbols and stuff
\newcommand{\CC}{\mathbb C}
\newcommand{\FF}{\mathbb F}
\newcommand{\NN}{\mathbb N}
\newcommand{\QQ}{\mathbb Q}
\newcommand{\RR}{\mathbb R}
\newcommand{\ZZ}{\mathbb Z}
\newcommand{\OO}{\mathcal{O}}
\renewcommand{\o}{\mathscr{O}}
\newcommand{\BB}{\mathcal{B}}
\newcommand{\cc}{\mathcal{C}}

% Statements
\newtheorem{theorem}{Theorem}[section]
\newtheorem{lemma}[theorem]{Lemma}
\newtheorem{proposition}[theorem]{Proposition}
\newtheorem{corollary}[theorem]{Corollary}
\newtheorem{definition}[theorem]{Definition}
\newtheorem{example}[theorem]{Example}

\usepackage{hyperref}

\title{Analysis Notes}
\author{Kerry Tarrant}
\date\today

\begin{document}
\maketitle
\tableofcontents{}


\chapter{Measure Theory}

\section{Preliminaries}
\textit{Note}: Understand the analogy between measuring "volume" and the concept of this section.

\textit{Note}: Every open set in $\mathbb{R}$ is a countable union of disjoint open intervals. Every open set in $\mathbb{R}^{d}$, $d\geq 2$, is almost the disjoint union of closed cubes. Almost means that only the boundaries of the cubes can overlap.


A \textbf{point} $x\in \RR^{d}$ consists of a $d$-tuple of real numbers
\begin{center}
$x=(x_1,x_2,\cdots,x_d), \hspace*{15pt} x_i\in \RR,$ for $i=1,\cdots,d$.
\end{center}

The \textbf{norm} of $x$ is denoted by $|x|$ and is defined to be the standard Euclidean norm given by
\begin{center}
    $|x|=(x_1^2+\cdots+x_d^2)^{1/2}$.
\end{center}

The \textbf{distance between two points} $x$ and $y$ is then simply $|x-y|$.

The \textbf{complement} of a set $E$ in $\RR^d$ is denoted by $E^c$ and defined by
\begin{center}
    $E^c=\{x\in\RR^d:x\notin E\}$.
\end{center}

The \textbf{distance between two sets} $E$ and $F$ is defined by
\begin{center}
    $d(E,F)=\inf |x-y|$,
\end{center}

where the infimum is taken over all $x\in e$ and $y\in F$.

The \textbf{open ball} in $\RR^d$ centered at $x$ and of radius $r$ is defined by
\begin{center}
    $B_r(x)=\{y\in\RR^d:|y-x|<r$.
\end{center}

A subset $E\subset\RR^d$ is \textbf{open} if for every $x\in E$ there exists $r>0$ with $B_r(x)\subset E$. A set is \textbf{closed} if its complement is open.

A point $x\in\RR^d$ is a \textbf{limit point} of the set $E$ if for every $r>0$, the ball $B_r(x)$ contains points of $E$.

An \textbf{isolated point} of $E$ is a point $x\in E$ such that there exists an $r>0$ where $B_r(x)\cap E$ is equal to $\{x\}$.

A closed set $E$ is \textbf{perfect} if $E$ does not have any isolated points.

A (closed) \textbf{rectangle} $R$ in $\RR^d$ is given by the product of $d$ one-dimensional closed and bounded intervals
\begin{center}
    $R=[a_1,b_1]\times [a_2,b_2]\times\cdots\times[a_d,b_d]$,
\end{center}
where $a_j\leq b_j$ are real numbers, $j=1,2,\dots,d$.

The \textbf{volume} of the rectangle $R$ is denoted by $|R|$, and is defined to be
\begin{center}
    $|R|=(b_1-a_1)\cdots(b_d-a_d)$.
\end{center}
A union of rectangles is said to be \textbf{almost disjoint} if the interiors are disjoint.

% Lemma 1.1 page 4
\begin{lemma}
    If a rectangle is the almost disjoint union of finitely many other rectangles, say $R=\cup_{k=1}^N R_k$, then
    \begin{center}
        $|R|=\sum_{k=1}^{N} |R_k|$.
    \end{center}
\end{lemma}

% Lemma 1.2 page 5
\begin{lemma}
    If $R,R_1,\dots,R_N$ are rectangles, and $R\subset\cup_{k=1}^N R_k$, then
    \begin{center}
        $|R|\leq\sum_{k=1}^N |R_k|$.
    \end{center}
\end{lemma}

% Theorem 1.3 page 5
\begin{theorem}
    Every open subset $\OO$ of $\RR$ can be written uniquely as a countable union of disjoint open intervals.
\end{theorem}

% Theorem 1.4 page 7
\begin{theorem}
    Every open subset $\OO$ of $\RR^d$, $d\geq1$, can be written as a countable union of almost disjoint closed cubes.
\end{theorem}

% The Cantor Set page 8
Construction of the \textbf{Cantor set}:
\begin{enumerate}
    \item Subintervals, starting at $[0,1]$, delete a third of the remaining closed intervals. This iterative process is performed countable many times. Example:
\begin{enumerate}
    \item[(0)] $C_0=[0,1]$
    \item[(1)] $C_1=[0,1/3]\cup[2/3,1]$
    \item[(2)] $C_2=[0,1/9]\cup[2/9,1/3]\cup[2/3,7/9]\cup[8/9,1]$
    
    \hspace{5pt}\vdots
    \item[(n)] $C_n$
     
    \hspace{5pt}\vdots 
\end{enumerate}
    \item This procedure yields a sequence of nested compact sets with
    \begin{center}
        $C_0\supset C_1\supset\cdots\supset C_k \supset C_{k+1}\supset\cdots$.
    \end{center}
    \item The \textbf{Cantor set} is defined by the intersection of all $C_k$'s:
    \begin{center}
        $\cc=\cap_{k=0}^{\infty}C_k$.
    \end{center}
\end{enumerate}

The Cantor set a length of zero.

$C_k$ is a disjoint union of $2^k$ intervals of length $3^{-k}$, making the total length of $C_k$ equal to $(2/3)^k$.

\section{The Exterior Measure}
\end{document}